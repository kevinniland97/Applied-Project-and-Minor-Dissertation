\chapter{Technology Review}
This chapter discusses the different technologies used in throughout the project. It discusses the the advantages and disadvantages of each technology and why certain technologies were used over others.

\section{Overview}
This project is comprised of React as the frontend, Flask as the server, MongoDB as the database, and ... to host the application. Throughout the project, the following technologies were also used and tested before the above was ultimately chosen:
\begin{itemize}
    \item Angular/Ionic
    \item Firebase
    \item Django
    \item MySQL
    \item Amazon Web Services
\end{itemize}

\section{Main Technologies}
This section will discuss the main technologies currently in use in the web application. The preceding section will discuss other technologies tried but ultimately weren't incorporated. 

\newpage

\subsection{React}
React (also known as React.js or ReactJS) is a JavaScript library for building user interfaces. It is maintained by Facebook and a community of individual developers and companies.

React can be used as a base in the development of single-page or mobile applications. However React is only concerned with rendering data to the DOM and so creating React applications usually requires the use of additional libraries for state management and routing. Redux and React Router are respective examples of such libraries. 

\subsubsection{Advantages}
React has many advantages, several of which apply to this project:

% https://www.altexsoft.com/blog/engineering/the-good-and-the-bad-of-reactjs-and-react-native/
\paragraph{Virtual DOM in ReactJS makes user experience better and developer’s work faster}

\paragraph{Redux: convenient state container}

\paragraph{Wide React and Redux toolset}

\subsubsection{Disadvantages}


\subsection{Flask}

\subsubsection{Advantages}

\subsubsection{Disadvantages}


\subsection{MongoDB}

\subsubsection{Advantages}

\subsubsection{Disadvantages}